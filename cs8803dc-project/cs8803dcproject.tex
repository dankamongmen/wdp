%-----------------------------------------------------------------------------
%
%               Template for sigplanconf LaTeX Class
%
% Name:         sigplanconf-template.tex
%
% Purpose:      A template for sigplanconf.cls, which is a LaTeX 2e class
%               file for SIGPLAN conference proceedings.
%
% Author:       Paul C. Anagnostopoulos
%               Windfall Software
%               978 371-2316
%               paul@windfall.com
%
% Created:      15 February 2005
%
%-----------------------------------------------------------------------------


\documentclass[]{sigplanconf}

% The following \documentclass options may be useful:
%
% 10pt          To set in 10-point type instead of 9-point.
% 11pt          To set in 11-point type instead of 9-point.
% authoryear    To obtain author/year citation style instead of numeric.

\usepackage{graphicx}
\usepackage{amsmath}

\begin{document}

\conferenceinfo{CS8803DC 2010}{May 6, Atlanta.} 
\copyrightyear{2010} 
\toappear{Copyright is held by the author.\\
\textit{CS8803DC} May 6, Atlanta.}
%\authorpermission
%\titlebanner{CS8803DC, Spring 2010, Georgia Institute of Technology}        % These are ignored unless
%\preprintfooter{Dynamic Translation for Intel's Loop Stream Decoder}   % 'preprint' option specified.

\title{Daytripper}
\subtitle{Dynamic Translation for Intel's Loop Stream Decoder}

\authorinfo{Nick Black}
           {nickblack@linux.com}

\maketitle

\begin{abstract}
Intel processors since the 65nm Conroe Core\texttrademark2
have included hardware to queue decoded loops directly into the out-of-order
execution engines\footnote{For an excellent comparison of the Loop Stream
Detector to the Pentium 4's trace cache, see\cite{kanter}.}. These ``Loop
Stream Detectors'' (LSD) allow for substantial power savings and, in some
cases, performance improvements. Unfortunately, the LSD can only store
instruction streams meeting a number of architecture-specific restrictions.
The Loop Stream Decoder represents an unmistakable power optimization, can be
definitively verified via the Core\texttrademark i7's LSD\_UOPS performance
counter, and has clearly-defined requirements for successful use. All these
properties make optimizing for the LSD an attractive prospect, especially for
a runtime translator.  
\end{abstract}

\category{D.3.4}{Processors}{Compilers}

\terms
Binary translation, instruction decoding

\keywords
Loop Stream Detector, Binary translation, Length-Changing Prefix, MSROM

\section{Introduction}
\cite{inteloptimize}

\section{Related work}

\appendix
\section{Using the LSD\_UOPS PMC}

Existing open source hardware profiling tools have yet to fully support the
Core\texttrademark i7 line of processors. The line represents several
combinations of Family, Model and Stepping numbers\cite{intelcpuid}.
\subsection{Linux's \texttt{perf}}
The \texttt{perf} tool, included in Linux source distributions since 2.6.31,
can support LSD\_UOPS via use of a ``raw counter''. Provide \texttt{-e -r 1a8}
when an event needs be specified; this selects unit mask 0x1 from event 0xa8,
the LSD\_UOPS event\cite{intelsys}.

\acks
Brought to you by Purple Drank\texttrademark. Code like a Champion Today!
\includegraphics[width=\columnwidth]{texobjs/drank.jpg}

% We recommend abbrvnat bibliography style.

\bibliographystyle{abbrvnat}
% The bibliography should be embedded for final submission.
\bibliography{cs8803dcproject}

\end{document}
