\documentclass[letterpaper,10pt]{article}
\usepackage{fullpage}
\usepackage{times}
\usepackage{cite}
\usepackage{fancyvrb}
\usepackage{moreverb}
\usepackage{graphicx}
\usepackage{hyperref}

\newcommand{\squishlist}{\begin{list}{$\bullet$}
  {\setlength{\itemsep}{0pt}
    \setlength{\parsep}{3pt}
    \setlength{\topsep}{3pt}
    \setlength{\partopsep}{0pt}
    \setlength{\leftmargin}{1.5em}
    \setlength{\labelwidth}{1em}
    \setlength{\labelsep}{0.5em}}}

\newcommand{\squishend}{\end{list}}

\title{CS8803SS Spring 2010 Project - GPU Security}
\author{
  Nick Black and Jason Rodzik
  % Should be alphabetical order
}
\date{}

\begin{document}
\maketitle

\section{Abstract}
Graphics processing units (GPUs) have advanced to a point where they are now capable of extremely complex generalized tasks in addition to their normal use for graphics processing.  The accessibility of GPUs to more generalized applications carries the side-effect of introducing a new avenue of security vulnerabilities.  While some research exists into the use of GPUs for general purpose and security techniques, our work breaks new ground by proposing techniques for attacking the GPU itself.

Blah blah here's some of the cool stuff we're doing.

\section{Introduction}
Heterogeneous computing has definitely arrived, and graphics processing units (GPUs) in the millions are employed worldwide. Tradition has shown newly programmable domains to be rapidly subjected (and often found vulnerable) to attacks of the past; indeed, wherever processing goes, so does the possibility of automated attacks. With high powered GPU's moving from the gamer's desktop to the laboratory's cluster, it's essential that the security issues surrounding their use be fleshed out earlier rather than later. Unfortunately, this is not the case; graphics card manufacturers have not publicized any of their own internal security studies (if any exist), and popular manufacturers are infamous for their
hostility to open source and academia\footnote{See for instance Wikipedia's page on
``\href{http://en.wikipedia.org/wiki/NVIDIA\_and\_FOSS}{Graphics hardware and FOSS}''.}
% This needs to be revised a bit and needs to have more added to it...
We have identified a variety of attack paths on existing GPU platforms from NVIDIA and ATI in addition to methods for countering these attacks.

\section{Our Work}
Blah Blah blah here's what we've done...

\section{Conclusion}
Let's restate everything we've already said in our abstract, introduction, and results...

\section{Related work}
Our research in this area is focused on the possibility of security
vulnerabilities against the GPU itself, an area which we were unable to find
prior research for. However, our research is motivated in part by an increase
in applications being developed for GPUs and also by GPUs being used as the
processing unit for security related tasks.
  
\subsection{General applications}
  
  Dedicated graphical processing units (GPUs) have seen significant advancement
in the past couple decades due to the insatiable demand that consumers have for
ever-increasing advances in video game graphics. This development has led to
recent GPUs reaching the point where they are powerful enough that there has
been significant research invested into using them to execute tasks outside the
realm of video games. One of the most popular examples of this is the
Folding@Home project, which in recent years has developed a version of its
application that runs on both ATI and nVidia brand GPUs
\url{http://folding.stanford.edu/English/Guide#ntoc4}.
  
  In 2007, nVidia released the SDK for CUDA, their "Compute Unified Device
Architecture," allowing for the GPU to be much more accessible to developers
wishing to use the GPU for non-graphical applications. In the three years since
then, a wide variety of applications for CUDA have been developed\footnote{\url{http://www.nvidia.com/object/what\_is\_cuda\_new.html}}.
  
  General Mills developed an application that used CUDA to model the optimal
way to cook a frozen pizza in a microwave. SeismicCity used CUDA to improve the
amount of time it takes to interpret seismic data to determine the optimal
drilling locations for finding oil
\url{http://www.nvidia.com/object/cuda\_in\_action.html}.
  
  Other companies are offering CUDA solutions for problems in the realms of
electromagnetics, bioinformatics, finance, accelerator physics, aerodynamics,
engine optimizations, image and video stream compression, healthcare and life
sciences, medical imaging, defense, and more
\squishlist
\item \url{http://www.acceleware.com/default/index.cfm/professional-services/}
\item \url{http://www.aneo.fr/index.php?option=com\_content&amp;view=article&amp;id=91&amp;Itemid=110}
\item \url{http://www.parallel-compute.com/}
\item \url{http://www.caps-entreprise.com/nvidia.html}
\item \url{http://www.elegant-mathematics.com/index.html?NVGPU}
\item \url{http://www.culatools.com/consulting}
\item \url{http://www.fixstars.com/en/solutions/gpu/}
\item \url{http://www.hoopoe-cloud.com/Services.aspx}
\item \url{http://www.hpc-project.com/gpu2.htm}
\item \url{http://www.sagivtech.com/21541.html}
\item \url{http://www.stoneridgetechnology.com/services/visualcomputing.asp}
\item \url{http://gpucomputing.txcorp.com/}.
\squishend
  
\subsection{Security applications}
  There are a few research areas where GPUs have been used in security applications, but all of these involve using the GPU as a faster processor than the CPU, and none of the research involves investigating the GPU itself or coding frameworks such as CUDA from a security standpoint.
  
  The most common security task that GPUs have been used for in prior work have been to use the GPU for performing cryptographic computations. Research has found that GPUs can perform some AES-related OpenSSL computations up to 20 times as fast as a typical implementation
\url{http://www.manavski.com/downloads/PID505889.pdf}. Another aspect that is of importance, particularly to copyright groups such as the RIAA and MPAA, are the encryption techniques used with GPUs for the security of applications involving remote displays, such as for HDCP blu-ray implementations\url{http://www.amazon.com/CryptoGraphics-Exploiting-Graphics-Security-Information/dp/038729015X}.
  
  While GPUs can be used for improving the speed of cipher implementations, they can be used to speed up cryptographic attacks as well. In some cases, applications have been developed using CUDA to allow for WiFi keys to be broken up to 100 times as fast as in typical implementations \url{http://arstechnica.com/security/news/2008/10/company-puts-nvida-gpus-to-work-cracking-wireless-security.ars}.
  
  Intrusion detection systems can exhibit performance degradation under heavy
  loads, and some of the proposed solutions to this problem have involved
  offloading IDS computations to the GPU
  \url{http://citeseerx.ist.psu.edu/viewdoc/download?doi=10.1.1.125.3302&amp;rep=rep1&amp;type=pdf}.
  Additionally, some techniques in pattern-matching are difficult to perform
  quickly enough to be useful, and GPUs have been proposed as a solution that
  allow for significant decreases in computation time
  \url{http://portal.acm.org/citation.cfm?id=1395492}.


\end{document}
