\documentclass[letterpaper,10pt]{article}
\usepackage{fullpage}
\usepackage{textcomp}
\usepackage{times}
\usepackage{cite}
\usepackage{fancyvrb}
\usepackage{moreverb}
\usepackage{graphicx}
\usepackage{hyperref}
\usepackage{multicol}

\newcommand{\squishlist}{\begin{list}{$\bullet$}
  {\setlength{\itemsep}{0pt}
    \setlength{\parsep}{3pt}
    \setlength{\topsep}{3pt}
    \setlength{\partopsep}{0pt}
    \setlength{\leftmargin}{1.5em}
    \setlength{\labelwidth}{1em}
    \setlength{\labelsep}{0.5em}
  } }

\newcommand{\squishend}{\end{list}}

\title{System-Centric Threat Modeling of CUDA\footnote{CUDA\textsuperscript{\textregistered}
is a registered trademark of the NVIDIA Corporation.}}
\author{Nick Black and Jason Rodzik\\
CS8803SS Project, Spring 2010}
\date{}

\begin{document}
\maketitle

\begin{abstract}
Within modern desktops, high-performance workstations, and even laptops there
exist sources of tremendous computing power: programmable, massively parallel graphics cards\footnote{Also known as ``unified shaders'' and General-Purpose GPUs (GPGPUs).}.
The hundreds of simple in-order cores found on recent NVIDIA and ATI hardware provide many more peak FLOPS than the x86 processor packages with
which they're commonly coupled.
Devices such as NVIDIA's Tesla\texttrademark{} have already traded video outputs for onboard memory, giving rise to ``desktop
supercomputing.'' Those who would fully utilize their machines can
no more ignore heterogeneous processors than multiple cores or SIMD\@. Leaving
large matrix multiplications on general-purpose, out-of-order CPUs is about as
efficient as eschewing hardware multipliers in favor of iterated addition.

It is, however, almost certainly safer. GPUs have traditionally accelerated
single-instance, interactive tasks (such as video games and CAD) and
decomposable, compute-bound tasks (such as batch rendering) for which a
deinterleaved schedule is optimal. Hardware facilities for the support of
multiple threads or processes are primitive at best. When the GPU was limited to
blitting, fogging and aliasing framebuffers, minimal isolation and
assurance capabilities were tolerable. Compromises of scientific data,
cryptographic materials and SCADA systems are rather more serious, yet these
precise applications are driving GPGPU's adoption. We provide a system-centric threat
model analysis of NVIDIA's CUDA system. We attack the system, drawing blood.
We close with software-level defenses against our attacks, and their limitations.
Our results suggest that, unless GPGPU systems add complexity related to security state (and thus devote
less transistor budget to performance), they cannot provide a trusted computing
base.

\end{abstract}

\begin{multicols}{2}
\section{Introduction}
Heterogeneous computing has definitely arrived, and graphics processing units (GPUs) in the millions are employed worldwide. Tradition has shown newly programmable domains to be rapidly subjected (and often found vulnerable) to attacks of the past; indeed, wherever processing goes, so does the possibility of automated attacks. With high powered GPU's moving from the gamer's desktop to the laboratory's cluster, it's essential that the security issues surrounding their use be fleshed out earlier rather than later. Unfortunately, this is not the case; graphics card manufacturers have not publicized any of their own internal security studies (if any exist), and popular manufacturers are infamous for their
hostility to open source and academia\footnote{See for instance Wikipedia's page on
``\href{http://en.wikipedia.org/wiki/NVIDIA\_and\_FOSS}{Graphics hardware and FOSS}''.}
% This needs to be revised a bit and needs to have more added to it...
We have identified a variety of attack paths on existing GPU platforms from NVIDIA and ATI in addition to methods for countering these attacks.

\section{An overview of CUDA}
CUDA is accessed via \texttt{open(2)}ing an appropriate device node and
invoking \texttt{ioctl(2)}\footnote{See Appendix~\ref{strace} for sample \texttt{strace(2)}
output.} upon the resulting file descriptor. All accesses to the card and its memory
originate from the (ring 0) driver; only the \texttt{ioctl(2)} calls themselves
are executed in userspace. The device node is set 0660 by default;
ability to make use of the GPU therefore depends upon membership in the
associated ``\texttt{video}'' group. Aside from this filesystem-level access check, no
isolation or authentication schemes are known to be present in NVIDIA's driver
or hardware. It ought be noted that failure to set a device file descriptor
close-on-exec via \texttt{O\_CLOEXEC} or \texttt{FD\_CLOEXEC} will
result in a usable device descriptor being duplicated into child processes,
and that these file descriptors can (by design or error) be passed via
\texttt{SCM\_RIGHTS} messages on \texttt{PF\_UNIX} sockets.

CUDA devices are either integrated or standalone; integrated hardware typically
claims a chunk of system RAM, and promptly marks it unprotected. No memory
protection hardware exists for onboard GDDR\@. All accesses to video memory go
through the card, unless page-locked memory is comutually mapped. In the
absence of shared mappings, data must be explicitly copied from video to system
memory (and vice versa) to be shared among computation devices\cite{cudaguide}.

CUDA kernels are typically distributed as JIT-friendly PTX binaries\cite{kerr},
an intermediate representation suitable for all NVIDIA hardware. Upon being
loaded onto the card, dynamic compilation\footnote{Likely performed in the
driver, not the hardware, though we have not yet verified this.} is performed,
resulting in a locally-optimized CUBIN blob. These blobs are dispatched to
Streaming Multiprocessors across the card, all of which share a common memory.
A given system thread can use only one device at a time, but a device may be
used by more than one system thread.

CUDA maintains forward compatibility, but augments the instruction set
between various ``Compute Capabilities''.  CUDA through Capability 1.3 uses
32-bit physical addressing, while Capability 2.0 uses 40 bits.

\section{Threat model}
Well-known threat taxonomies include the ``CIA Triad'' (extended by the
``Parkerian Hexad\cite{sechandbook}'') and Microsoft's STRIDE\@. The latter, developed as part of
Microsoft's Secure Product Lifecycle, is designed for system-centric threat
modeling and suitable for our purposes. We assume that an attacker has access
to the NVIDIA device node(s) (by default, \texttt{/dev/nvidiaX}); such privileges
are necessary to compute on the device, and thus a safe assumption. Escalation
to device access is an attack on the operating system's access control, and
outside the scope of this paper.

We seek to answer the following questions, for machines with one or more CUDA-capable
cards:
\subsection{Spoofing}
\squishlist
\item Is it possible for one CUDA kernel to pre\"empt data copies requested
from another?
\squishend
\subsection{Tampering}
\squishlist
\item Can one CUDA kernel manipulate another's data set?
\item Is it possible to manipulate the system's video channel from an arbitrary
CUDA kernel? If so, how might this be further leveraged (for instance, replacing
a browser padlock icon or disabling system notifications)? What about textures?
\item Is it possible to construct a debugging environment around other CUDA kernels?
\squishend
\subsection{Repudiation}
\squishlist
\item Can a CUDA kernel disassociate itself from the system process which spawned it?
\item Can a CUDA kernel spawn new CUDA kernels?
\item What forensic data, if any, is created as a result of CUDA computing?
\squishend
\subsection{Information disclosure}
\squishlist
\item Can a CUDA kernel read another kernel's data set? Need they be simultaneously
scheduled for this to occur?
\item Is it possible to read another CUDA kernel's code, even if it cannot be
controlled?
\item Is it possible to read the system memory of another CUDA application via
calls through the CUDA intermediary?
\item Is it possible to reconstruct the system's video channel from an arbitrary
CUDA kernel? What about textures?
\squishend
\subsection{Denial of service}
\squishlist
\item Can a CUDA kernel monopolize resources in the face of competitors?
\item Can a CUDA kernel prevent another from being controlled, or executing
data transfers to or from the system?
\item Can a CUDA kernel deny resources beyond the GPU\@?
\squishend
\subsection{Escalation of privilege}
\squishlist
\item Might the driver be exploited, allowing arbitrary ring 0 code to run?
\item If it is possible to return doppelg\"anger data, might it be leveraged
to attack (and hopefully exploit) system-side processes?
\item Is it possible to arbitrarily (i.e.\ without exploitation) manipulate
another CUDA kernel's code? Is it possible to construct a CUDA virus?
\item Is it possible to arbitrarily (i.e.\ without exploitation) manipulate a
system process's code maps from a CUDA kernel?
\squishend

\section{Methodology}
It is first necessary to develop a model of memory protection, multiple user
contexts, and memory layout in CUDA. Few details have been made public
regarding these topics; what public knowledge exists takes the form of
scattered fora posts, Wikis\cite{nouveaucuda}, and mailing lists. Foremost among these is memory
protection; a trusted multiprocess computing base cannot be constructed in its
absence. Memory protection for host accesses of the GPU could be implemented at
three levels, each more effective than the last:
\squishlist
\item the proprietary CUDA and OpenGL libaries,
\item the proprietary \texttt{nvidia.ko} kernel driver, and
\item on the hardware itself.
\squishend
Userspace protection can likely be thwarted by userspace code, whereas
protection implemented within the kernel module ought be secure against
all but ring 0 operations (recall that our threat model does not assume access
to ring 0). It is unlikely that protection on the card itself can be generally
circumvented. Furthermore, this is the only place to protect memory from CUDA
kernels.

We implemented the \texttt{cudadump} and
\texttt{cudascorch} tools (see below)%fixme use ref%
to explore memory protection.

\subsection{Tools}
We developed
\squishlist
\item \texttt{cudadump} to read raw address ranges, generating as output a
bitmap of valid and invalid accesses, and an optional disk image.
\item \texttt{cudascorch} to write raw address ranges, generating as output a
bitmap of valid and invalid accesses.
\squishend
We made use of
\squishlist
\item \texttt{cuda-gdb}\cite{cudagdb} to debug CUDA programs,
\item \texttt{strace}\cite{stracecode} to track system calls,
\item \texttt{nv50dis}\cite{nv50dis} to disassemble nv50 binaries,
\item \texttt{nvtrace}\cite{nvtrace} to decipher \texttt{ioctl(2)}s
issued to the NVIDIA proprietary driver, and
\item The Linux kernel's MMIOTrace\cite{mmiotrace}
infrastructure to track memory-mapped I/O operations.
\squishend

\section{Attacking CUDA}
Some addressing details are definitively declared in the PTX Reference\cite{ptxguide}.
Through Compute Capability 1.3, load and store instructions require a "state space"
modifier in addition to an address. General-purpose registers (\texttt{r0}--\texttt{r127}) are
indexed via a 7-bit immediate. Global memory, accessed via one of 16 linear ranges
of up to 4GB each (\texttt{g0}--\texttt{g15}), is addressed via the
contents of a 32-bit general purpose register. Other state spaces are addressed
via the sum of an address register (\texttt{a0}--\texttt{a7}) and an immediate
offset. Constant memory is referenced through one of 16 linear ranges of up to
64KB each (\texttt{c0}--\texttt{c15}), as is a block-shared region of up to 16KB
(\texttt{s}). The per-thread local memory (\texttt{l}), an abstraction atop the
global memory, performs address translation based on the block and thread indices.
Platforms such as the Tesla\textsuperscript{\texttrademark} provide the full 
4GB of memory accessible via these 32 bits.

Compute Capability 2.0 unifies addressing, but provides backwards-compatibility.
Virtual addresses are extended to 64 bits, with 40 physical bits currently
supported for up to 1TB of RAM.
\section{Defending CUDA}

\section{Conclusions} 
The recommendations of the Orange Book\cite{orangebook}, requirements of the Common Criteria\cite{ccrit},
and Saltzer-Schroeder\cite{principles} have long provided general principles for
security-conscious design. Any number of these principles have been ignored
(or poorly effected) along the way to NVIDIA's and ATI's current GPUs. These
manufacturers focused nigh-exclusively on performance relative to cost and
power, resulting in some of the world's most powerful devices. Harnessing these
mighty processing units is necessary to even approach full machine utilization.
This will become only more true:
\begin{itemize}
\item Increasing the clock frequency affects cooling and power requirements
(dynamic power consumption is proportional to frequency). Furthermore,
clock frequency increases tend to force decomposition of pipeline
stages, exacerbating branch misprediction delays and pressuring any OOO
system\cite{cormean}.
\item Increasing the issue width, or duplicating functional units, either
requires ISA changes (for VLIW) or massive frontend resources for lookahead,
disambiguation, and hazard tracking. Assuming these expenses acceptable,
diminishing returns result from limitations of instruction-level parallelism.
Furthermore, very wide issues lead to sparse code flows, taxing the instruction
store subsystem. 
\item Further investments in cache --- and thus, hopefully, fewer memory
delays --- serve only to approach more closely a device's theoretical peak.
That peak itself is a function of architecture, not of programs or their
access patterns.
\item Investments in OOO apparatus (reorder buffers,
frontend reservation stations, register renaming) yield diminishing returns due
to the profound limitations of instruction level parallelism\cite{phenn}. Like improvements
to cache, OOO can only hide delays, not find new FLOPS\@.
\item Denser chip-multithreading similarly serves only to hide latency.
\item Larger chips require either pipelined wires or high voltages to operate
reliably, and place strict requirements on clock-signaling circuity. Efficient
power management requires complex and expensive partial power gating\footnote{For
instance, feeding $\mu$ops from Nehalem's Loop Stream Detector results in power-down
of the leading three pipeline stages.}.
\end{itemize}
We see that FLOPS --- at any price --- can be had only by adding cores or extending
SIMD order. The former is obviously easier with small, in-order, SISD cores,
such as the unified shaders of a modern GPU\@; duplicating a Nehalem core is a
much more daunting process. As for the latter, note that Intel's ``Sandy
Bridge'' microarchitecture is expected to add the AVX instruction scheme and
its 256-bit \texttt{YMM} vector registers\cite{avx}\cite{fma4}. The data-based decomposition of CUDA's
``SIMT'' model can immediately take advantage of their new cores (given
sufficiently large problem sets, of course), whereas x86 binaries\footnote{Most of them, anyway.} would
require dynamic translation or recompilation to take advantage of the new SIMD
order.

Extensive manycoring and huge memory bandwidths are certainly the key to FLOPS,
but programmable GPUs in their current state are damningly insecure. Without the
addition of infrastructure and policy, they must not be trusted in a
multiprocessing context. Even in the absence of calculated attack, one's own
mistakes or accidents can wreck undetected havoc. It ought be noted that memory
protection could be implemented without the added complexity of
hardware-supported virtual memory (provided some assistance from the OS); this
would immediately address most issues.

\section{Related work}
Our research in this area is focused on the possibility of security
vulnerabilities against the GPU itself, an area which we were unable to find
prior research for. However, our research is motivated in part by an increase
in applications being developed for GPUs and also by GPUs being used as the
processing unit for security related tasks.
  
\subsection{General applications}
  
  Dedicated graphical processing units (GPUs) have seen significant advancement
in the past couple decades due to the insatiable demand that consumers have for
ever-increasing advances in video game graphics. This development has led to
recent GPUs reaching the point where they are powerful enough that there has
been significant research invested into using them to execute tasks outside the
realm of video games. One of the most popular examples of this is the
Folding@Home project, which in recent years has developed a version of its
application that runs on both ATI and nVidia brand GPUs
\url{http://folding.stanford.edu/English/Guide#ntoc4}.
  
  In 2007, nVidia released an SDK for CUDA, their ``Compute Unified Device
Architecture'', allowing for the GPU to be much more accessible to developers
wishing to use the GPU for non-graphical applications. In the three years since
then, a wide variety of applications for CUDA have been developed\footnote{\url{http://www.nvidia.com/object/what\_is\_cuda\_new.html}}.
  
  General Mills developed an application that used CUDA to model the optimal
way to cook a frozen pizza in a microwave. SeismicCity used CUDA to improve the
amount of time it takes to interpret seismic data to determine the optimal
drilling locations for finding oil
\url{http://www.nvidia.com/object/cuda\_in\_action.html}.
  
  Other companies are offering CUDA solutions for problems in the realms of
electromagnetics, bioinformatics, finance, accelerator physics, aerodynamics,
engine optimizations, image and video stream compression, healthcare and life
sciences, medical imaging, defense, and more
\squishlist
\item \url{http://www.acceleware.com/default/index.cfm/professional-services/}
\item \url{http://www.aneo.fr/index.php?option=com\_content&amp;view=article&amp;id=91&amp;Itemid=110}
\item \url{http://www.parallel-compute.com/}
\item \url{http://www.caps-entreprise.com/nvidia.html}
\item \url{http://www.elegant-mathematics.com/index.html?NVGPU}
\item \url{http://www.culatools.com/consulting}
\item \url{http://www.fixstars.com/en/solutions/gpu/}
\item \url{http://www.hoopoe-cloud.com/Services.aspx}
\item \url{http://www.hpc-project.com/gpu2.htm}
\item \url{http://www.sagivtech.com/21541.html}
\item \url{http://www.stoneridgetechnology.com/services/visualcomputing.asp}
\item \url{http://gpucomputing.txcorp.com/}.
\squishend
  
\subsection{Security applications}
  There are a few research areas where GPUs have been used in security applications, but all of these involve using the GPU as a faster processor than the CPU, and none of the research involves investigating the GPU itself or coding frameworks such as CUDA from a security standpoint.
  
  The most common security task that GPUs have been used for in prior work have been to use the GPU for performing cryptographic computations. Research has found that GPUs can perform some AES-related OpenSSL computations up to 20 times as fast as a typical implementation
\url{http://www.manavski.com/downloads/PID505889.pdf}. Another aspect that is of importance, particularly to copyright groups such as the RIAA and MPAA, are the encryption techniques used with GPUs for the security of applications involving remote displays, such as for HDCP blu-ray implementations\url{http://www.amazon.com/CryptoGraphics-Exploiting-Graphics-Security-Information/dp/038729015X}.
  
  While GPUs can be used for improving the speed of cipher implementations, they can be used to speed up cryptographic attacks as well. In some cases, applications have been developed using CUDA to allow for WiFi keys to be broken up to 100 times as fast as in typical implementations \url{http://arstechnica.com/security/news/2008/10/company-puts-nvida-gpus-to-work-cracking-wireless-security.ars}.
  
  Intrusion detection systems can exhibit performance degradation under heavy
  loads, and some of the proposed solutions to this problem have involved
  offloading IDS computations to the GPU
  \url{http://citeseerx.ist.psu.edu/viewdoc/download?doi=10.1.1.125.3302&amp;rep=rep1&amp;type=pdf}.
  Additionally, some techniques in pattern-matching are difficult to perform
  quickly enough to be useful, and GPUs have been proposed as a solution that
  allow for significant decreases in computation time
  \url{http://portal.acm.org/citation.cfm?id=1395492}.

\bibliographystyle{acm}
\bibliography{cs8803ssProject}
\end{multicols}
\appendix
\newpage
\section{\texttt{strace(2)d} CUDA binary}\label{strace}
\VerbatimInput[fontsize=\scriptsize,numbers=left,framerule=.8mm,label=strace(2)d deviceQueryDrv]{texobjs/deviceQueryDrv.strace}
\end{document}
