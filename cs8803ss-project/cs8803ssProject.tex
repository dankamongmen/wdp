\documentclass[letterpaper,10pt]{article}
\usepackage{fullpage}
\usepackage{textcomp}
\usepackage{times}
\usepackage{cite}
\usepackage{fancyvrb}
\usepackage{moreverb}
\usepackage{graphicx}
\usepackage{hyperref}
\usepackage{multicol}
\usepackage{verbatim}

\newcommand{\squishlist}{\begin{list}{$\bullet$}
  {\setlength{\itemsep}{0pt}
    \setlength{\parsep}{3pt}
    \setlength{\topsep}{3pt}
    \setlength{\partopsep}{0pt}
    \setlength{\leftmargin}{1.5em}
    \setlength{\labelwidth}{1em}
    \setlength{\labelsep}{0.5em}
  } }

\newcommand{\squishend}{\end{list}}

\title{My Other Computer is your GPU:\\
System-Centric CUDA Threat Modeling with CUBAR}
\author{Nick Black and Jason Rodzik\\
CS8803SS Project, Spring 2010}
\date{}

\begin{document}
\maketitle

\begin{quotation}\scriptsize\textit{``The charm of history and its enigmatic lesson consist in the fact that, from age to age, nothing changes and yet everything is completely different.''}

\hfill---Aldous Huxley \end{quotation}

\begin{abstract}
\textbf{Heterogeneous computing has definitely arrived, and graphics processing
units (GPUs) in the millions are employed worldwide. History has shown newly
programmable domains to be rapidly subjected (and often found vulnerable) to
attacks of the past. We enumerate a system-centric threat model using Microsoft's
STRIDE process\cite{stride}. We then describe an active general purpose
programming system, NVIDIA's Compute Unified Device Architecture
(CUDA) \footnote{NVIDIA, CUDA, and GeForce are trademarks or registered trademarks of the NVIDIA Corporation.},
and explore the threat space. We derive
and describe previously-undisclosed memory protection and translation processes
in CUDA, successfully mount several attacks on the CUDA system, and identify
numerous directions for future work. Our CUBAR suite of tools, especially \texttt{cudash}
(the CUDA Shell), form a solid platform for research to come.}

\end{abstract}
\begin{multicols}{2}
\section{Introduction}
Within modern desktops, high-performance workstations, and even laptops there
exist sources of tremendous computing power: programmable, massively parallel graphics cards.
The hundreds of simple in-order cores found on recent NVIDIA and ATI hardware provide many more peak FLOPS than the x86 processor packages with
which they're commonly coupled.
Devices such as NVIDIA's Tesla\texttrademark{} have already traded video outputs for onboard memory, giving rise to ``desktop
supercomputing.'' Those who would fully utilize their machines can
no more ignore heterogeneous processing than multiple cores or SIMD\@.

It is, however, almost certainly safer. GPUs have traditionally accelerated
single-instance, interactive tasks (such as video games and CAD) and
decomposable, compute-bound tasks (such as batch rendering) for which a
deinterleaved schedule is optimal. Hardware facilities for the support of
multiple threads or processes are primitive at best. When the GPU was limited to
blitting, fogging, and anti-aliasing framebuffers, minimal isolation and
assurance capabilities were tolerable. Compromises of scientific data,
cryptographic materials and SCADA systems are rather more serious, yet these
precise applications drive the adoption of GPGPU programming. 

We explored CUDA 3.0 on a 2.6.34-rc2 Linux kernel, using version 195.36.15
of the CUDA software. Hardware included a GTS 360M, a GeForce 8400 GS, and a
GT 9600. Our source code can be found on GitHub\footnote{\url{http://github.com/dankamongmen/wdp/tree/master/cs8803ss-project/}}.

\section{An overview of CUDA}
A GPU-accelerated CUDA application on Linux adds four component parts to
a typical process:
\squishlist
\item A ``fat binary'' containing host and guest object code, typically x86 and PTX\cite{ptxguide}.
Note that PTX is a mere intermediate representation, converted to the native
CUBIN format at runtime.
\item The CUDA userspace library \texttt{libcuda.so}. This closed-source archive
provides an interface to the NVIDIA driver, and management of a CUDA instance.
\item The NVIDIA driver \texttt{nvidia.ko}. This closed-source module handles
\texttt{ioctl}s issued by \texttt{libcuda.so}.
\item The hardware, controlled by registers and memory-mapped IO.
\squishend
No complete open-source CUDA solution yet exists.

The CUDA language is an extension of C++, designed for compilation with NVIDIA's \texttt{nvcc}\cite{nvcc}
toolchain. NVIDIA states that a given device supports multiple host threads,
but that a given host thread can control only one device. Data must either be
explicitly copied between system and video RAM or, on devices of Compute Capability
1.2 or higher, restricted to shared memory maps\cite{cudaguide}. GPU processes
(``kernels'') can be launched asynchronously, though (until Compute Capability
2.0) only one kernel can be executed at a time by a given device.

CUDA supports numerous memory ``spaces'', selected via PTX affixes (until the
advent of unified addressing in Compute Capability 2.0, presumably effected via
memory translation). Spaces differ according to caching, method of
initialization, visibility across the device's multiprocessors, and mutability.
Official documentation does not address memory protection or translation.

CUDA kernels are typically distributed as JIT-friendly PTX binaries\cite{kerr},
an intermediate representation suitable for all NVIDIA hardware. Upon being
loaded onto the card, dynamic compilation\footnote{Likely performed in the
driver, not the hardware, though we have not yet verified this.} is performed,
resulting in a locally-optimized CUBIN blob. These blobs are dispatched to
Streaming Multiprocessors across the card, all of which share a common memory.
A given system thread can use only one device at a time, but a device may be
used by more than one system thread.

In our case, the \texttt{nvidia.ko} kernel module and \texttt{libcuda.so}
library weigh in at thirteen and seven megabytes respectively. We can assume
them to contain substantial logic. Ultimately, some of our questions can be
answered only via analysis of these binaries. This is a matter of some importance
for open source projects seeking to duplicate CUDA functionality, such as the
Nouveau Project and \texttt{libcudest}\cite{libcudest}.

\subsection{Existing controls}
The \texttt{/dev/nvidiaX} device nodes control access to the CUDA hardware and
kernelspace component. Under the standard Linux security model, these
will be restricted via 0660 permissions to a group (typically \texttt{video}).
Membership in the owning group is thus necessary and sufficient to satisfy the
operating system. \texttt{strace} output for CUDA applications, along with
NVIDIA's \texttt{nvidia-smi} device configuration program, show use of the
\texttt{geteuid} system call, but no further access controls are exported to
the user.

\section{Threat model}
Well-known threat taxonomies include the ``CIA Triad'' (extended by the
``Parkerian Hexad\cite{sechandbook}'') and Microsoft's STRIDE\@. The latter, developed as part of
Microsoft's Secure Product Lifecycle, is designed for system-centric threat
modeling and suitable for our purposes. We assume that an attacker has access
to the NVIDIA device node(s) (by default, \texttt{/dev/nvidiaX}); such privileges
are necessary to compute on the device, and thus a safe assumption. Escalation
to device access is an attack on the operating system's access control, and
outside the scope of this paper.

We seek to answer the following questions, for machines with one or more CUDA-capable
cards:
\subsection{Spoofing}
\squishlist
\item Is it possible for one CUDA kernel to pre\"empt data copies requested
from another?
\squishend
\subsection{Tampering}
\squishlist
\item Can one CUDA kernel manipulate another's data set?
\item Is it possible to construct a debugging environment around other CUDA kernels?
\squishend
\subsubsection{Video Tampering}
\squishlist
\item Can a CUDA kernel modify the active display?
\item Can a CUDA kernel recover the active display?
\item What unique issues arise from such capabilities?
\squishend
\subsection{Repudiation}
\squishlist
\item Can a CUDA kernel disassociate itself from the system process which spawned it?
\item Can a CUDA kernel spawn new CUDA kernels?
\item What forensic data, if any, is created as a result of CUDA computing?
\squishend
\subsection{Information disclosure}
\squishlist
\item Can a CUDA kernel read another kernel's data set? Need they be simultaneously
scheduled for this to occur?
\item Is it possible to read another CUDA kernel's code, even if it cannot be
controlled?
\item Is it possible to read the system memory of another CUDA application via
calls through the CUDA intermediary?
\item Is it possible to reconstruct the system's video channel from an arbitrary
CUDA kernel? What about textures?
\squishend
\subsection{Denial of service}
\squishlist
\item Can a CUDA kernel monopolize resources in the face of competitors?
\item Can a CUDA kernel prevent another from being controlled, or executing
data transfers to or from the system?
\item Can a CUDA kernel deny resources beyond the GPU\@?
\squishend
\subsection{Escalation of privilege}
\squishlist
\item Might the driver be exploited, allowing arbitrary ring 0 code to run?
\item If it is possible to return doppelg\"anger data, might it be leveraged
to attack (and hopefully exploit) system-side processes?
\item Is it possible to arbitrarily (i.e.\ without exploitation) manipulate
another CUDA kernel's code? Is it possible to construct a CUDA virus?
\item Is it possible to arbitrarily (i.e.\ without exploitation) manipulate a
system process's code maps from a CUDA kernel?
\squishend

\section{Methodology}
It is first necessary to develop a model of memory protection, multiple user
contexts, and memory layout in CUDA. Few details have been made public
regarding these topics; what public knowledge exists takes the form of
scattered fora posts, Wikis\cite{nouveaucuda}, and mailing lists. Foremost among these is memory
protection; a trusted multiprocess computing base cannot be constructed in its
absence. Memory protection for host accesses of the GPU could be implemented at
three levels, each more effective than the last:
\squishlist
\item the proprietary CUDA and OpenGL libaries,
\item the proprietary \texttt{nvidia.ko} kernel driver, and
\item on the hardware itself.
\squishend
Userspace protection can likely be thwarted by userspace code, whereas
protection implemented within the kernel module ought be secure against
all but ring 0 operations (recall that our threat model does not assume access
to ring 0). It is unlikely that protection on the card itself can be generally
circumvented. Furthermore, this is the only place to protect memory from CUDA
kernels.

We implemented the \texttt{cudadump} and
\texttt{cudascorch} tools (see below)%fixme use ref%
to explore memory protection.

\subsection{Tools}
We developed
\squishlist
\item \texttt{cudadump} (and its helper binary, \texttt{cudaranger}) to
discover readable regions in a given virtual memory,
\item \texttt{cudash}, the CUDA Shell, to perform close-in experiments and
prototype attacks, and
\item \texttt{cudabounder}, \texttt{cudaminimal}, \texttt{cudapinner}, \texttt{cudaquirky},
\texttt{cudaspawner}, and \texttt{cudastuffer}, a series of single-purpose attack tools.
\squishend
We made use of
\squishlist
\item \texttt{vbtracetool}\cite{vbtrace} to dump VIDEO BIOS,
\item \texttt{cuda-gdb}\cite{cudagdb} to debug CUDA programs,
\item \texttt{strace}\cite{stracecode} to track system calls,
\item \texttt{nv50dis}\cite{nv50dis} to disassemble nv50 binaries,
\item \texttt{nvtrace}\cite{nvtrace} to decipher \texttt{ioctl}s issued to the NVIDIA proprietary driver, and
\item the Linux kernel's MMIOTrace\cite{mmiotrace} infrastructure to track memory-mapped I/O operations.
\squishend

\section{Attacking CUDA}
Some addressing details are definitively declared in the PTX Reference.
Through Compute Capability 1.3, load and store instructions require a "state space"
modifier in addition to an address. General-purpose registers (\texttt{r0}--\texttt{r127}) are
indexed via a 7-bit immediate. Global memory, accessed via one of 16 linear ranges
of up to 4GB each (\texttt{g0}--\texttt{g15}), is addressed via the
contents of a 32-bit general purpose register. Other state spaces are addressed
via the sum of an address register (\texttt{a0}--\texttt{a7}) and an immediate
offset. Constant memory is referenced through one of 16 linear ranges of up to
64KB each (\texttt{c0}--\texttt{c15}), as is a block-shared region of up to 16KB
(\texttt{s}). The per-thread local memory (\texttt{l}), an abstraction atop the
global memory, performs address translation based on the block and thread indices.
Platforms such as the Tesla\textsuperscript{\texttrademark} provide the full 
4GB of memory accessible via these 32 bits.

Compute Capability 2.0 unifies addressing, but provides backwards-compatibility.
Virtual addresses are extended to 64 bits, with 40 physical bits currently
supported for up to 1TB of RAM.

\begin{comment}
One of the possibilities we investigated was whether an arbitrary CUDA kernel would be able 
to modify the system's video channel.  The threat model we considered was arbitrary code
being able to manipulate the video channel.  

In one case, this could consist of grabbing the current 
framebuffer to obtain information from other programs.  In another case, this could consist of modifying the 
current framebuffer in an effort to deceive the user into believing the system is in a different state than 
it actually is.  For example, if one could replace the padlock icon
commonly used on browsers to indicate a secure web connection, an attacker could fool a user into believing 
that they have a secure and authentic connection when in fact they do not.

The memory that CUDA code can access consists of three types: per-thread local memory, 
per-block shared memory, and global memory.  However, this memory differs from the framebuffer 
that is used by the graphics card to update the display.  We analyzed the possibility of both reading 
and writing to the framebuffer in the case of CUDA applications.

We first examined the case of reading the current framebuffer.  Not only is this necessary in order 
to be able to modify it, but it can lead to information disclosure as well.  While CUDA provides several 
methods for obtaining the current framebuffer, all of these methods rely on interfacing with OpenGL or Direct3D 
through the provided interfaces.  

In the case of OpenGL, glReadPixels() can be used to obtain the current framebuffer.  
While this can be done within a CUDA program, it is not done using CUDA itself.  The data can be pulled into CUDA directly 
via several methods, including the use of cudaGraphicsMapResources or cudaGLMapBufferObject\cite{cudaguide}\cite{cudareference}.  In this instance, CUDA provides no added benefit or significant difference over existing programs.

Similarly, writing to the framebuffer is not explicitely possible in CUDA, and similar methods must be used as to read from it (i.e. using OpenGL or Direct3D within CUDA).

However, CUDA provides both read and write access to video memory, and the framebuffer is contained within memory.  Unfortunately, our tests were unsuccessful at locating the location of the framebuffer in memory.  This indicates either that the CUDA API disallows access to the region of memory containing the framebuffer or that there is a system in place that provides a hardware barrier between the framebuffer and the rest of VRAM for this purpose.

%WTF happened on my system - artifacting.  Could this be exploited somehow?%
While we were unable to specifically locate the framebuffer in memory, during our tests we did encounter several cases of video artifacting on our displays.  This led us to believe that there must be some method of accessing the framebuffer by using CUDA in a way that it was not intended for.  As of the writing of this paper we have been unable to isolate the circumstances that led to the artifacting and to reproduce them in a controlled manner, but we believe there is promise in future work being done in this area.

%Hopefully the above section can be updated, but as of now I have no clue what happened on my system to produce the artifacting.  I'll take another look at it but it'll probably have to wait until after my take-home final is done%
\end{comment}

\section{Defending CUDA}

\section{Other Vendors}
Examining graphics architectures of other vendors was outside the scope of our research.  NVIDIA's leading competitor, ATI, provides a similar framework, ATI Stream\footnote{\url{http://www.amd.com/US/PRODUCTS/TECHNOLOGIES/STREAM-TECHNOLOGY/Pages/stream-technology.aspx}}.  We feel that similar security issues have been overlooked in ATI's model as well given the lack of security discussion in this area.  Additionally, the high-performance and cut-throat nature of the graphics card market has likely precluded any thoughts on security implementations.  We believe that if ATI Stream was examined with the same threat model in mind it is certainly possible that similar security concerns would come to light.

\section{Related work}
Our research in this area is focused on the possibility of security
vulnerabilities against the GPU itself, an area which we were unable to find
prior research for. However, our research is motivated in part by an increase
in applications being developed for GPUs and also by GPUs being used as the
processing unit for security related tasks.
  
\subsection{General applications}
  
  Dedicated graphical processing units (GPUs) have seen significant advancement
in the past couple decades due to the insatiable demand that consumers have for
ever-increasing advances in video game graphics. This development has led to
recent GPUs reaching the point where they are powerful enough that there has
been significant research invested into using them to execute tasks outside the
realm of video games. One of the most popular examples of this is the
Folding@Home project, which in recent years has developed a version of its
application that runs on both ATI and NVIDIA brand GPUs\cite{foldingathome}.
  
  In 2007, NVIDIA released an SDK for CUDA, their ``Compute Unified Device
Architecture'', allowing for the GPU to be much more accessible to developers
wishing to use the GPU for non-graphical applications. In the three years since
then, a wide variety of applications for CUDA have been developed\footnote{\url{http://www.nvidia.com/object/what\_is\_cuda\_new.html}}.
  
  General Mills developed an application that used CUDA to model the optimal
way to cook a frozen pizza in a microwave. SeismicCity used CUDA to improve the
amount of time it takes to interpret seismic data to determine the optimal
drilling locations for finding oil\cite{cudainaction}.
  
  Other companies are offering CUDA solutions for problems in the realms of
electromagnetics, bioinformatics, finance, accelerator physics, aerodynamics,
engine optimizations, image and video stream compression, healthcare and life
sciences, medical imaging, defense, and more\footnote{For a full list, see \url{http://www.nvidia.com/object/cuda\_in\_action.html}}.
  
\subsection{Security-related computation}
  There are a few research areas where GPUs have been used in security applications, but all of these involve using the GPU as a faster processor than the CPU, and none of the research involves investigating the GPU itself or coding frameworks such as CUDA from a security standpoint.
  
  The most common security task that GPUs have been used for in prior work have been to use the GPU for performing cryptographic computations. Research has found that GPUs can perform some AES-related OpenSSL computations up to 20 times as fast as a typical implementation
\cite{cudaaes}. Another aspect that is of importance, particularly to copyright groups such as the RIAA and MPAA, are the encryption techniques used with GPUs for the security of applications involving remote displays, such as for HDCP blu-ray implementations\cite{cryptographics}.
  
  While GPUs can be used for improving the speed of cipher implementations, they can be used to speed up cryptographic attacks as well. In some cases, applications have been developed using CUDA to allow for WiFi keys to be broken up to 100 times as fast as in typical implementations \footnote{\url{http://www.elcomsoft.com/edpr.html}}.
  
  Intrusion detection systems can exhibit performance degradation under heavy
  loads, and some of the proposed solutions to this problem have involved
  offloading IDS computations to the GPU
  \cite{offloadids}.
  Additionally, some techniques in pattern-matching are difficult to perform
  quickly enough to be useful, and GPUs have been proposed as a solution that
  allow for significant decreases in computation time
  \cite{gpuids}.

\subsection{GPU security}
 There were a few instances of prior art that actually consider the idea of security on the GPU.  An implementation of a wrapper library for the CUDA SDK features the use of memory-scrubbing for the specific purpoes of wiping the GPU memory after use to provide security from subsequent users\cite{cudawrapper}.  This indicates that while it has not been a subject of much discussion in literature, there are some developers that are starting to consider the security implications of developing applications for GPUs.
 
 During the course of our research we also found a debian bug report for a severe security problem in nvidia-kernel-common\cite{bugreport}.  The security implications of the bug were severe: each user registered with consolekit automatically receives read/write access for every device on the system.  As this applies to devices such as /dev/sda, it constituted a serious security problem.  While programming errors resulting in security vulnerabilities are nothing new, we believe this provides yet another example of how security must be on the forefront of designers' and developers' minds whenever new avenues are explored.

 \section{Conclusions} 
The recommendations of the Orange Book\cite{orangebook}, requirements of the Common Criteria\cite{ccrit},
and animadversions of Saltzer and Schroeder\cite{principles} have long provided general principles for
security-conscious design. Any number of these principles have been ignored
(or poorly effected) along the way to NVIDIA's and ATI's current GPUs. These
manufacturers focused nigh-exclusively on performance relative to cost and
power, resulting in some of the world's most powerful devices. Harnessing these
mighty processing units is necessary to even approach full machine utilization.
This will become only more true:
\begin{itemize}
\item Increasing the clock frequency affects cooling and power requirements
(dynamic power consumption is proportional to frequency). Furthermore,
clock frequency increases tend to force decomposition of pipeline
stages, exacerbating branch misprediction delays and pressuring any OOO
system\cite{cormean}.
\item Increasing the issue width, or duplicating functional units, either
requires ISA changes (for VLIW) or massive frontend resources for lookahead,
disambiguation, and hazard tracking. Assuming these expenses acceptable,
diminishing returns result from limitations of instruction-level parallelism.
Furthermore, very wide issues lead to sparse code flows, taxing the instruction
store subsystem. 
\item Further investments in cache --- and thus, hopefully, fewer memory
delays --- serve only to approach more closely a device's theoretical peak.
That peak itself is a function of architecture, not of programs or their
access patterns.
\item Investments in OOO apparatus (reorder buffers,
frontend reservation stations, register renaming) yield diminishing returns due
to the profound limitations of instruction level parallelism\cite{phenn}. Like improvements
to cache, OOO can only hide delays, not find new FLOPS\@.
\item Denser chip-multithreading similarly serves only to hide latency.
\item Larger chips require either pipelined wires or high voltages to operate
reliably, and place strict requirements on clock-signaling circuity. Efficient
power management requires complex and expensive partial power gating\footnote{For
instance, feeding $\mu$ops from Nehalem's Loop Stream Detector results in power-down
of the leading three pipeline stages.}.
\end{itemize}
We see that FLOPS --- at any price --- can be had only by adding cores or extending
SIMD order. The former is obviously easier with small, in-order, SISD cores,
such as the unified shaders of a modern GPU\@; duplicating a Nehalem core is a
much more daunting process. As for the latter, note that Intel's ``Sandy
Bridge'' microarchitecture is expected to add the AVX instruction scheme and
its 256-bit \texttt{YMM} vector registers\cite{avx}\cite{fma4}. The data-based decomposition of CUDA's
``SIMT'' model can immediately take advantage of their new cores (given
sufficiently large problem sets, of course), whereas x86 binaries\footnote{Most of them, anyway.} would
require dynamic translation or recompilation to take advantage of the new SIMD
order.

Extensive manycoring and huge memory bandwidths are certainly the key to FLOPS,
but programmable GPUs in their current state are damningly insecure. Without the
addition of infrastructure and policy, they must not be trusted in a
multiprocessing context. Even in the absence of calculated attack, one's own
mistakes or accidents can wreck undetected havoc. It ought be noted that memory
protection could be implemented without the added complexity of
hardware-supported virtual memory (provided some assistance from the OS); this
would immediately address most issues.

\bibliographystyle{acm}
\bibliography{cs8803ssProject}
\end{multicols}
\appendix
\newpage
\section{\texttt{strace(2)d} CUDA binary}\label{strace}
\VerbatimInput[fontsize=\scriptsize,numbers=left,framerule=.8mm,label=strace(2)d deviceQueryDrv]{texobjs/deviceQueryDrv.strace}
\newpage
\section{\texttt{libcuda.so.195.36.15} 3.0 symbols}\label{strace}
\VerbatimInput[fontsize=\scriptsize,numbers=left,framerule=.8mm,label=dynamic symbols]{texobjs/cudasymbols}
\section{\texttt{dmesg} output from hung task}\label{oops}
\VerbatimInput[fontsize=\scriptsize,numbers=left,framerule=.8mm,label=dmesg]{texobjs/crash}
\end{document}
