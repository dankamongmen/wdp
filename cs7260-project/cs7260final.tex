\documentclass[letterpaper,10pt]{article}
\usepackage{hyperref}
\usepackage{amsmath,amsthm,amssymb}
\usepackage[scriptsize,bf]{caption}
\usepackage{fullpage}
\usepackage{textcomp}
\usepackage{times}
\usepackage{cite}
\usepackage{fancyvrb}
\usepackage{moreverb}
\usepackage{graphicx}
\usepackage{multicol}

\makeatletter
\newenvironment{tablehere}
{\def\@captype{table}}
{}
\newenvironment{figurehere}
{\def\@captype{figure}}
{}
\makeatother

\newcommand{\squishlist}{\begin{list}{$\bullet$}
  {\setlength{\itemsep}{0pt}
    \setlength{\parsep}{3pt}
    \setlength{\topsep}{3pt}
    \setlength{\partopsep}{0pt}
    \setlength{\leftmargin}{1.5em}
    \setlength{\labelwidth}{1em}
    \setlength{\labelsep}{0.5em}
  } }

\newcommand{\squishend}{\end{list}}

\title{LRUMAP: $O(1)$ LRU for Classifying Network Probes}
\author{Nick Black and Jolene Tarosky\\
CS7260 Project, Spring 2010}
\date{}

\begin{document}
\maketitle

\begin{abstract}
Network administrators, particularly in proactively secure environments such
as military installations, banks and SCADA sites, seek to discriminate standard
network accesses from reconnaissance. In both host-based intrusion prevention
systems (such as Solar Designer's \texttt{scanlogd}\footnote{\url{http://www.openwall.com/scanlogd/}})
and network-level applications (such as the open source Snort\footnote{\url{http://www.snort.org/}} IDS),
this is accomplished via comparing against some threshold the ratio of a remote
host's successful accesses of the network to unsuccessful accesses. While a
successful access will almost inevitably elicit some response packet, a
reconnaissance may well receive no reply. Classification ought thus not be performed
at reply time, but only upon a potential probe's ingress.

First-packet classification requires predicting the success of a network access.
Especially among environments heterogeneous in ownership or architecture (data centers, research facilities,
development firms, etc), the likelihood of a given service (identified by host
and transport addresses, and network protocol) existing can vary widely from
host to host. Tracking these services dynamically is a natural fit for LRU, but
traditional $O(n)$ or even $O(\lg{n})$ LRU implementations are too slow for
backbone routers and IPS devices. We introduce LRUSET, a $O(1)$ true LRU scheme
which becomes more space-efficient as the number of monitored hosts increases.
We then illustrate several other counting problems from networking algorithmics
which could be benefited by this novel data structure.
\end{abstract}

\begin{multicols}{2}
\section{Introduction}
\textbf{WRITE ME! -- JOLENE}
\textbf{can you also tighten up the abstract maybe?}
\cite{varghese}
\cite{xu}

\section{Conventions}
We will speak of $n$ independent \textit{sets}. It is assumed that all possible
input values are somehow partitioned among these sets. For a set of TCP/UDP
endpoint pairs, a set might be associated with each observed IP address pair
(perhaps themselves backed by a one- or two-leveled LRU). For an associative
cache, some subset of the address bits are used to index into a fixed number of
sets. In an \textit{order-$r$ LRU}, each set contains $r$ members. When $r=1$,
the system is said to be \textit{direct-mapped}: replacement always occurs, and
the number of sets $n$ is equivalent to the system capacity. When $r$ equals
the capacity of the system, there can likewise be at most one set; such a system
is said to be \textit{fully associative}. Since $\lg{r}$ is a frequent term in
LRU's complexity analysis, a system's order is almost always a power of 2\footnote{Though
wasteful of space, other orders have been infrequently used to improve latency\cite{intelcpuid}.}.
\section{LRUMAP Data Structures}
At LRUMAP's core lies a transition table shared among all sets, its size
dependent upon $r$. This transition table is initialized at startup, and can be
placed into constant memory once assembled (alternatively, it can be cast into
cheap hardware using extremely reliable Masked-OR ROM (MROM)
technology\cite{ice}). By \textit{transition table}, we mean a family of $|P|$
well-defined functions \begin{equation*}
\sigma_{1}(l),\dotsc,\sigma_{p}(l)\longrightarrow{P}
\end{equation*} where $p\in{P}$ and $1\le{l}\le{r}$. $P$ is the set of all
permutations of $r$ integers, and has $r!$ members. As it requires $\lceil\lg{r!}\rceil$
bits to encode $p$, the total size of this table is $r!r\lceil\lg{r!}\rceil$ bits.

True LRU assigns to each member in each set an index $l$, $0\le{l}\le{r-1}$.
Each set thus requires $\lg{r}$ bits to represent the relative ages of its $r$
members. The central insight behind LRUMAP is that these members have been
placed on a bijection against $0\dotsc{r-1}$; this is the classic definition of
a \textit{permutation} mapping. Each set's metastate can thus be considered
completely described by a permutation of $r$, and $\lg{r!}$ bits are obviously
sufficient to identify a set's LRU state. Just as in classic LRU, metastate has
a variable cost linearly dependent on $n$, but each set as a whole maps into
the precomputed permutation table using $\lceil\lg{r!}\rceil$ bits. It is
simple to prove that the latter is strictly less than the former by properties
of logarithms:
\begin{align*}
\lg{ab} &= \lg{a} + \lg{b} \\
\lg{r!} &= \lg{(1*2*\dotsb*r-1*r)} \\
\lg{r!} &= \sum_{i=1}^{r}{\lg{i}} \\
&= \sum_{i=1}^{r}{\lg{i}} \le \sum_{i=1}^{r}{\lg{r}}
\end{align*}
\section{LRUMAP Algorithms}
\textbf{WRITE ME! -- NICK}
\section{Comparison to Classic LRU}
The essential time and space complexities of classic LRU and LRUMAP are
captured in Table ~\ref{tab:lru}:
\begin{center}
\begin{tablehere}
	\begin{tabular}{|l|l|l|}
	\hline
	& LRU & LRUMAP \\
	\hline
	Time & $O(\lceil\frac{r}{p}\rceil), {p}\le{r}$ & O(1) \\
	\hline
	Space & $O(nr\lg{r})$ & $O(r!r\lceil\lg{r!}\rceil) + O(n\lceil\lg{r!}\rceil)$ \\
	\hline
	\end{tabular}
	\caption{Essential properties of LRU/LRUMAP}
	\label{tab:lru}
\end{tablehere}
\end{center}
As expected, we see that LRUMAP saves significant space over LRU for large
values of $n$ and small values of $r$. At eighth order, LRUMAP takes over
from LRU at $n=111217$:
\begin{figurehere}
	\centering
	\includegraphics[width=\columnwidth]{out/lrumap8.pdf}
	%\caption{Eighth-order LRU}
\end{figurehere}
LRUMAP is even more effective at fourth order, requiring less space than LRU
for all but trivially small $n$:
\begin{figurehere}
	\centering
	\includegraphics[width=\columnwidth]{out/lrumap4.pdf}
	%\caption{Fourth-order LRU}
\end{figurehere}
\section{Beyond LRUMAP}
\subsection{Optimizations}
It may not be necessary to perform true, precise LRU. An entire family of
schemes have been presented, approximating LRU in less time and/or space.
Intel\cite{shanley} and Via processors of the Pentium era made use of
\textit{pseudo-LRU}, a direction vector-based scheme which requires only
$O(\lg{r})$ bits per set\cite{handy}. The PA-RISC 8600\cite{hurd} likewise used
a proprietary \textit{quasi-LRU} algorithm, with a similar reduction in bits
per set. These schemes can be straightforwardly combined with LRUMAP to yield a
new variant, wherein the LRUMAP entries represent and map among these
algorithms' direction vectors rather than an order's permutations. There seems
no advantage in doing so, however; Pseudo-LRUMAP will consume strictly more space
than Pseudo-LRU, and is unlikely to provide a speed advantage.

A direction vector is $\lg{r}$ bits, and the set of direction vectors is thus
composed of $2^{\lg{r}}=r$ members. Just as before, we precompute a constant
table, this time containing $r \lg{r}$-bit transitions for each of $r$
entries. Each of $n$ sets will require a $\lg{r}$-bit encoding of its current
pseudo-LRU state. A single operation still suffices to update the metastate.
Again allowing for $p$ updates in parallel:
\begin{center}
\begin{tablehere}
	\begin{tabular}{|l|l|l|}
	\hline
	& Pseudo-LRU & Pseudo-LRUMAP \\
	\hline
	Time & $O(\lceil\frac{\lg{r}}{p}\rceil), {p}\le{\lg{r}}$ & O(1) \\
	\hline
	Space & $O(n\lg{r})$ & $O(r^{2}\lg{r}) + O(n\lg{r})$ \\
	\hline
	\end{tabular}
	\caption{Essential properties of Pseudo-LRU/LRUMAP}
	\label{tab:pseudolru}
\end{tablehere}
\end{center}
In this case, however, the updates being performed involve $\lg{r}$ single bits.
We've assumed $r$ to be less than or equal to 8; even an 8-bit embedded processor
could thus perform the updates in parallel. Pseudo-LRUMAP cannot be expected to
provide any improvement over Pseudo-LRU.
\subsection{Extensions}
It is trivial to adapt LRUMAP to the Most-Recently-Used methodology, employed
by page caches when a ``looping sequential''\cite{dewitt} access pattern is
detected.
\section{Related Work}
\textbf{WRITE ME! -- JOLENE}
\bibliographystyle{acm}
\bibliography{cs7260final}
\end{multicols}
\end{document}
