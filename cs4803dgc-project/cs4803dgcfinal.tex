\documentclass[letterpaper,10pt]{article}
\usepackage{fullpage}
\usepackage{textcomp}
\usepackage{times}
\usepackage{cite}
\usepackage{fancyvrb}
\usepackage{moreverb}
\usepackage{graphicx}
\usepackage{hyperref}
\usepackage{multicol}

\newcommand{\squishlist}{\begin{list}{$\bullet$}
  {\setlength{\itemsep}{0pt}
    \setlength{\parsep}{3pt}
    \setlength{\topsep}{3pt}
    \setlength{\partopsep}{0pt}
    \setlength{\leftmargin}{1.5em}
    \setlength{\labelwidth}{1em}
    \setlength{\labelsep}{0.5em}
  } }

\newcommand{\squishend}{\end{list}}

\title{Freeing CUDA}
\author{Nick Black\\
Project Proposal, CS4803DGC Spring 2010}
\date{}

\begin{document}
\maketitle

\begin{abstract}
For several years, the \texttt{nouveau} project (\url{http://nouveau.freedesktop.org/wiki/})
has heroically wrested an Open Source, cleanroom implementation of 2D drivers for
NVIDIA cards\cite{nouveaustatus}. Early in 2010, RedHat Fedora Linux began driving
NVIDIA devices with \texttt{nouveau}. Shortly thereafter, NVIDIA dropped
support for their (feature-limited, shrouded-source) \texttt{nv} X.Org driver\footnote{The
closed-source \texttt{nvidia} driver is still actively supported.}. \texttt{Nouveau} represents
the clear and inevitable future of NVIDIA support on the Linux/X.Org system.

The Nouveau Project's primary aim is to provide a high-quality rendering
solution to the Linux desktop. Support for GPGPU has not been a priority, but
the path to Open Source GPGPU undoubtedly lies through
\texttt{Nouveau}'s advanced memory management infrastructure and well-designed hardware
abstractions, especially if it is to coexist with hardware-accelerated desktops.
As a believer in both the practical and ethical value of Open Source, I believe
it high time Linux's GPGPU stack was Freed, and that replacing \texttt{libcuda.so}
is the most immediately critical task.
\end{abstract}

\begin{multicols}{2}
\section{Introduction}
etc

\bibliographystyle{acm}
\bibliography{cs4803dgcfinal}
\end{multicols}
\appendix
appendix
\newpage
\end{document}
